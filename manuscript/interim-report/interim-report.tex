\documentclass{article}

\usepackage[preprint]{neurips_2024}
\usepackage[utf8]{inputenc}
\usepackage[T1]{fontenc}
\usepackage{hyperref}
\usepackage{url}
\usepackage{booktabs}
\usepackage{amsfonts}
\usepackage{nicefrac}
\usepackage{microtype}
\usepackage{xcolor}
\usepackage{graphicx}
\usepackage{algorithm}
\usepackage{algpseudocode}
\usepackage{amsmath}


\title{Differential Privacy in Image Classification using ResNet-20 and DP-SGD, DP-Adam, and DP-RMSProp Optimization techniques}

\author{
    Praveen Rangavajhula\\
    Department of Computer Science\\
    University of Georgia\\
    Athens, GA, 30602\\
    \texttt{praveen.rangavajhula@uga.edu} \\
    \And
    Alexander Darwiche\\
    Department of Computer Science\\
    University of Georgia\\
    Athens, GA, 30605 \\
    \texttt{alexander.darwiche@uga.edu} \\
    \And
    Deven Allen\\
    Department of Computer Science\\
    University of Georgia\\
    Athens, GA, 30605 \\
    \texttt{dca09692@uga.edu} \\
}

\begin{document}

    \maketitle
    \begin{abstract}
    test

    \end{abstract}


    \section{Introduction}\label{sec:introduction}
    
    Stochastic Gradient Descent (SGD) and its differentially private (DP) relative DP-SGD are frequently used optimizers for image classification tasks. DP-SGD
    introduces noise and gradient clipping to the standard SGD algorithm, to ensure that the underlying data remains private upon release of model parameters/hyperparameters. While DP-SGD
    is perhaps the most popular optimizer for tasks concerned with differential privacy, we also explore 2 additional optimization techniques. In this interim report, 
    we looked to explore the impact of changing the optimizer on the test accuracy achieveable on the CIFAR-10 dataset.
  
    The first additional optimizers tested in this paper are Differentially Private Root Mean Square Propogation (DP-RMSProp) and Adaptive Moment Estimation (DP-Adam). The motivation to try these optimizers is to properly bound
    the benefit of employing different optimization techniques from DP-SGD. RMSProp improves on SGD in that it includes a moving average of squared gradients. This additional term attempts to lessen the possibility of the
    vanishing/exploding gradient phenomenon. Adam, similarly, includes an estimation of first and second moment of gradients. Adam attempts to build on and improve on both RMSProp and AdaGrad \cite{kingma2017adammethodstochasticoptimization}.

    While implementing these additional optimizers is not novel unto itself, we believe it provides a solid groundwork for additional novelty and testing going forward. This interim report will
    highlight the current results of our testing and the anticipated path forward, given our now full implementation of DP-SGD, DP-RMSProp, and DP-Adam.

    \section{Formal Description of Models Tried}\label{sec:models}
    % \subsection{Motivation and Rationale}\label{subsec:motivation-and-rationale}

As mentioned earlier in the report, the motivation for trying these optimizers was to appropriately bound the strength of the 3 optimization techniques for CIFAR-10. 
This lays a strong groundwork for future additions to the algorithms and allows us to have a wide range of options in making these adjustments. 

DP-SGD is likely the most common optimization algorithm used in differentially private image classification. The motivation for using DP-SGD is simply to have a strong 
baseline for comparison. Most literature regarding differential privacy begins by using DP-SGD, so it would make sense as the first algorithm implemented and tested.

RMSProp and Adam can be viewed as successors to DP-SGD. Both introduce approaches that allow for adaptivity of the step size taken at each iteration of the algorithm. These algorithms
were first proposed to deal with the vanishing/exlpoding gradient phenomenon and have now been used frequently in the differentially private image classification literature. We felt that
implementing these algorithms would provide us with a strong bound on the performance benefits associated with changing optimizer, essentially treating optimizers as another hyperparameter
that we could tune.

\subsection{Description of Optimization Algorithm (Psuedocode)}\label{subsec:algorithm-description}

Below are pseudocodes for the DP-SGD, DP-RMSprop, and DP-Adam algorithms that we plan to privatize, adapted or referenced from~\cite{DBLP:journals/corr/abs-1807-06766}:
\begin{algorithm}
    \caption{DP-SGD}
    \label{alg:sgd}
    \begin{algorithmic}[1]
        \State \textbf{Input:} A step size $\alpha$, initial starting point $\mathbf{x}_1 \in \mathbb{R}^d$,
        and access to a (possibly noisy) oracle for gradients of $f : \mathbb{R}^d \rightarrow \mathbb{R}$.
        \Function{SGD}{$\mathbf{x}_1, \alpha$}
            \State Initialize: $\mathbf{v}_0 = \mathbf{0}$
            \For{$t = 1, 2, \dots$}
                \State \textbf{Compute gradient}
                \State $\mathbf{g}_t = \nabla f(\mathbf{x}_t)$
                \State \textbf{Clip gradient}
                \State $\mathbf{g}_t = \mathbf{g}_t/max(1, \frac{\lVert \mathbf{g}_t \rVert_{2} }{C})$
                \State \textbf{Add Noise}
                \State $\hat{\mathbf{g}_t} = \mathbf{g}_t + \mathcal{N}(0,\sigma^{2}C^{2}I)$
                \State \textbf{Descent}
                \State $\mathbf{x}_{t+1} = \mathbf{x}_t - \alpha \hat{\mathbf{g}_t}$
            \EndFor
        \EndFunction
    \end{algorithmic}
\end{algorithm}
\vspace{-1cm}

\begin{algorithm}
    \caption{DP-RMSProp}
    \label{alg:rmsprop}
    \begin{algorithmic}[1]
        \State \textbf{Input:} A constant vector $\mathbb{R}^d \ni \xi \mathbf{1}_d \geq 0$, parameter $\beta_2 \in [0, 1)$, step size $\alpha$, initial starting point $\mathbf{x}_1 \in \mathbb{R}^d$, and access to a (possibly noisy) oracle for gradients of $f : \mathbb{R}^d \rightarrow \mathbb{R}$.
        \Function{RMSProp}{$\mathbf{x}_1, \beta_2, \alpha, \xi$}
            \State Initialize: $\mathbf{v}_0 = \mathbf{0}$
            \For{$t = 1, 2, \dots$}
                \State \textbf{Compute gradient}
                \State $\mathbf{g}_t = \nabla f(\mathbf{x}_t)$
                \State \textbf{Clip gradient}
                \State $\mathbf{g}_t = \mathbf{g}_t/max(1, \frac{\lVert \mathbf{g}_t \rVert_{2} }{C})$
                \State \textbf{Add Noise}
                \State $\hat{\mathbf{g}_t} = \mathbf{g}_t + \mathcal{N}(0,\sigma^{2}C^{2}I)$
                \State \textbf{Calculate Moving Average of Squared Gradient}
                \State $\mathbf{v}_t = \beta_2 \mathbf{v}_{t-1} + (1 - \beta_2)(\hat{\mathbf{g}_t}^2 + \xi \mathbf{1}_d)$
                \State $\mathbf{V}_t = \text{diag}(\mathbf{v}_t)$
                \State \textbf{Descent}
                \State $\mathbf{x}_{t+1} = \mathbf{x}_t - \alpha \mathbf{V}_t^{-\frac{1}{2}} \hat{\mathbf{g}_t}$
            \EndFor
        \EndFunction
    \end{algorithmic}
\end{algorithm}
\vspace{-1cm}

\begin{algorithm}
    \caption{DP-Adam}
    \label{alg:adam}
    \begin{algorithmic}[1]
        \State \textbf{Input:} A constant vector $\mathbb{R}^d \ni \xi \mathbf{1}_d > 0$, parameters $\beta_1, \beta_2 \in [0, 1)$, a sequence of step sizes $\{\alpha_t\}_{t=1,2,\dots}$, initial starting point $\mathbf{x}_1 \in \mathbb{R}^d$, and oracle access to the gradients of $f : \mathbb{R}^d \to \mathbb{R}$.
        \Function{ADAM}{$\mathbf{x}_1, \beta_1, \beta_2, \{\alpha_t\}, \xi$}
            \State Initialize: $\mathbf{m}_0 = \mathbf{0}$, $\mathbf{v}_0 = \mathbf{0}$
            \For{$t = 1, 2, \dots$}
            \State \textbf{Compute gradient}
                \State $\mathbf{g}_t = \nabla f(\mathbf{x}_t)$
                \State \textbf{Clip gradient}
                \State $\mathbf{g}_t = \mathbf{g}_t/max(1, \frac{\lVert \mathbf{g}_t \rVert_{2} }{C})$
                \State \textbf{Add Noise}
                \State $\hat{\mathbf{g}_t} = \mathbf{g}_t + \mathcal{N}(0,\sigma^{2}C^{2}I)$
                \State \textbf{Calculate Moving Average of Squared Gradient}
                \State $\mathbf{v}_t = \beta_2 \mathbf{v}_{t-1} + (1 - \beta_2)(\hat{\mathbf{g}_t}^2 + \xi \mathbf{1}_d)$
                \State $\mathbf{V}_t = \text{diag}(\mathbf{v}_t)$
                \State \textbf{Calculate Moving Average of Gradient}
                \State $\mathbf{m}_t = \beta_1 \mathbf{m}_{t-1} + (1 - \beta_1) \hat{\mathbf{g}_t}$
                \State \textbf{Descent}
                \State $\mathbf{x}_{t+1} = \mathbf{x}_t - \alpha_t \left( \mathbf{V}_t^{\frac{1}{2}} + \operatorname{diag}(\xi \mathbf{1}_d) \right)^{-1} \mathbf{m}_t$

            \EndFor
        \EndFunction
    \end{algorithmic}
\end{algorithm}

\subsection{Modification and Variations}\label{subsec:modification-and-variations}


\subsection{Privacy Proof}\label{subsec:incremental-improvements}



    \section{Related Work}\label{sec:related-work}

    \break
    \section{Preliminary Results}\label{sec:prelim-results}
    % \subsection{Optimizer Details (with Hyperparameter values)}\label{subsec:optimizer-details}

\subsection{DP Details (Budget Accounting, Noise Scale, # Iterations)}\label{subsec:dp-details}
\begin{itemize}
    \item \textbf{Budget Accounting:} 
    \item \textbf{Noise Scale:}
    \item \textbf{Number of Iterations:}
\end{itemize}

 
    \break
    \section{Discussion of Results}\label{sec:results-discussion}
    % \subsection{Train/Test Loss/Accuracy with varied Epochs}\label{subsec:train-testloss-accuracy}
\begin{itemize}
    \item \textbf{Accuracy/Training Loss:} Accuracy and training loss on CIFAR-10~\cite{cifar10_dataset}.
    \item \textbf{Privacy Cost:} We will measure $(\epsilon, \delta)$ using RDP~\cite{Mironov_2017_RenyiDP} to ensure privacy compliance.
    \item \textbf{Training Time:} Time complexity and memory usage will be tracked.
\end{itemize}

\subsection{Ablation Study}\label{subsec:ablation-study}

\subsection{Graphs showing effects of Hyperparameters}\label{subsec:graphs-hyperparamters}

\subsection{Tables summarizing results (with standard deviation)}\label{subsec:summary-table-results}


    \break
    \section{Learned and Plans}\label{sec:learned-and-plans}
    

    \bibliographystyle{plain}
    \bibliography{references}


    \section*{GitHub Contributions}
    The code and related materials for this project are available at our GitHub repository:
    \url{https://github.com/CS8960-Privacy-Preserving-Data-Analysis/final-project}.
    Contributions, issues, and discussions are welcome.


\end{document}
