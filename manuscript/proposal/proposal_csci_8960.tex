\documentclass{article}

\usepackage[preprint]{neurips_2024}
\usepackage[utf8]{inputenc}
\usepackage[T1]{fontenc}
\usepackage{hyperref}
% \usepackage{url}
\usepackage{booktabs}
\usepackage{amsfonts}
\usepackage{nicefrac}
\usepackage{microtype}
\usepackage{xcolor}
\usepackage{graphicx}
\usepackage{algorithm}
\usepackage{algpseudocode}
\usepackage{amsmath}
\usepackage{array}
\usepackage{multirow}
\usepackage{pdflscape}
\usepackage{appendix}
\usepackage{adjustbox}



\title{Differential Privacy in Image Classification using ResNet-20 and DP-SGD Optimization}

\author{
    Praveen Rangavajhula\\
    Department of Computer Science\\
    University of Georgia\\
    Athens, GA, 30602\\
    \texttt{praveen.rangavajhula@uga.edu} \\
    \And
    Alexander Darwiche\\
    Department of Computer Science\\
    University of Georgia\\
    Athens, GA, 30605 \\
    \texttt{alexander.darwiche@uga.edu} \\
    \And
    Deven Allen\\
    Department of Computer Science\\
    University of Georgia\\
    Athens, GA, 30605 \\
    \texttt{dca09692@uga.edu} \\
}

\begin{document}

    \maketitle

    \begin{abstract}

        This project proposes a differentially private image classification system using ResNet-20 with various
        optimizers, starting with Differentially Private Stochastic Gradient Descent (DP-SGD) as a baseline.
        We aim to make incremental improvements with additional optimization techniques, exploring both non-private
        and DP versions of optimizers, and justifying our choices based on prior work and their potential
        to outperform others.
        The project will focus on achieving competitive accuracy while satisfying privacy guarantees.
        Additionally, we are investigating ways to enhance accuracy by modifying optimizer components,
        such as gradient clipping (potentially using techniques like automatic clipping).

    \end{abstract}


    \section{Introduction}\label{sec:introduction}

    The increase in prevalence of machine learning models, especially in image classification,
    has coincided with concerns over privacy~\cite{papernot2022hyperparametertuningrenyidifferential}.
    Differential privacy (DP) specifically addresses these concerns by ensuring that models do not inadvertently leak sensitive
    information about individual data points.
    In this proposal, we will use the ResNet-20 model, which is well-suited for datasets like CIFAR-10~\cite{Idelbayev_ResNet20},
    and will implement and improve DP-SGD to achieve better privacy guarantees without significantly compromising accuracy.


    \section{Motivation and Problem Statement}\label{sec:motivation-and-problem-statement}
    In many real-world applications, the concerns over privacy leakage can hinder the deployment of machine
    learning models.
    Current state-of-the-art models like ResNet-20 achieve high accuracy but are vulnerable to attacks that could leak
    sensitive information.
    An example of a differentially private optimizer is Differentially Private Stochastic Gradient Descent (DP-SGD), which has been shown
    to effectively reduce privacy leaks, but has challenges in balancing model accuracy and privacy~\cite{Abadi_2016_DeepLearningDifferentialPrivacy}.
    Even with the strong foundation that DP-SGD provides, we believe there is room for improvements that can
    achieve both higher accuracy and stronger privacy guarantees.



    \section{Methodology}\label{sec:methodology}
    \subsection{Model Architecture: ResNet-20}\label{subsec:model-architecture:-resnet-20}


We propose utilizing the ResNet-20 model~\cite{Idelbayev_ResNet20} for CIFAR-10,
a standard dataset for image classification tasks.
We selected a 20-layer ResNet for its deep architecture and strength in image classification problems~\cite{DBLP:journals/corr/HeZRS15}.
20
layers should be enough depth to adequately model many features, while not encountering the higher training error encountered
with excessively \("\)deep\("\) architectures. \cite{DBLP:journals/corr/HeZRS15}

If necessary, we may modify the architecture slightly to optimize for DP compatibility.

\subsection{Non-private Optimizers to Try}\label{subsec:non-private-optimizers-to-try}
We propose implementing 4 non-private optimizers to establish baseline performance for ResNet-20 on CIFAR-10.
Three of these optimizers will be First-Order optimizers
that all build on one another.
The last optimizer that we will explore is a Second-Order optimization technique known as Cubic .

\begin{itemize}
    \item \textbf{SGD:} Standard Stochastic Gradient Descent (SGD) for baseline comparison.
    This optimizer works by calculating the gradient at each data
    point and updating the model parameters with the following update rule:

    \[
        \mathbf{x}_1 = \mathbf{x}_1 - \alpha \cdot \mathbf{g}_t
    \]

    where $\mathbf{x}_1$ is model parameters, $\alpha$ is
    the learning rate, and $\mathbf{g}_t$ is the gradient at that data point.

    \item \textbf{RMSprop:} Root Mean Square Propagation (RMSprop) builds on SGD by including the \("\)moving average\("\) factor.
    This factor functions by scaling the gradient
    each step, based on the gradient of the previous data points.
    This is done by scaling the gradient, at each model parameter update,
    by the moving average squared gradient:

    \[
        \mathbf{v}_t = \beta_2 \mathbf{v}_{t-1} + (1 - \beta_2)(\mathbf{g}_t^2 + \xi \mathbf{1}_d)
    \]

    where $\mathbf{v}_{t-1}$ is the squared gradient average from the previous step, $\mathbf{g}_t^2 $ is
    the squared gradient of the current step, $\beta_2$ is the \("\)squared gradient moving average\("\) factor
    and $\xi \mathbf{1}_d$ is a constant vector~\cite{DBLP:journals/corr/abs-1807-06766,Jason_Huang_2020}.

    \item \textbf{ADAM:} Adaptive Moment Estimation (ADAM) further build on RMSprop by including another \("\)moving average\("\) factor, this time for the gradient.
    In the general gradient update rule formula, instead of gradient,
    ADAM substitutes in the gradient moving average:

    \[
        \mathbf{m}_t = \beta_1 \mathbf{m}_{t-1} + (1 - \beta_1) \mathbf{g}_t
    \]

    where $\mathbf{m}_{t-1}$ is the gradient moving average from the previous step, $\mathbf{g}_t$ is
    the gradient of the current step, and $\beta_1$ is the \("\)gradient moving average\("\) factor~\cite{DBLP:journals/corr/abs-1807-06766}.

\end{itemize}

\subsection{Differentially Private Optimizer: DP-SGD}\label{subsec:differentially-private-optimizer:-dp-sgd}
We propose implementing DP-SGD as our privacy-preserving algorithm.
The key components of DP-SGD are:
\begin{itemize}
    \item \textbf{Gradient Clipping:} Limits the influence of individual examples during training.
    \item \textbf{Noise Addition:} Adds noise to gradients to ensure privacy (via Opacus)~\cite{opacus}.
    \item \textbf{Privacy Accounting:} We will use Rényi Differential Privacy (RDP) for privacy budget tracking~\cite{Mironov_2017_RenyiDP}.
\end{itemize}

\subsection{Incremental Improvements}\label{subsec:incremental-improvements}
After we have privatized SGD, we propose making the following enhancements:
\begin{itemize}
    \item Adding an adaptive learning rate by incorporating the moving average of gradients squared (RMSprop)~\cite{DBLP:journals/corr/abs-1807-06766}
    \item Adding a moving average for gradient (ADAM)~\cite{DBLP:journals/corr/abs-1807-06766}
    \item Automatic gradient clipping methods that adjusts the clipping threshold throughout training~\cite{bu2023automaticclippingdifferentiallyprivate}.
\end{itemize}

\subsection{Rationale for Choosing DP-SGD}\label{subsec:rationale-for-choosing-dp-sgd}
DP-SGD provides well-documented privacy guarantees~\cite{Abadi_2016_DeepLearningDifferentialPrivacy}
while maintaining decent utility for image classification tasks.
The addition of noise and gradient clipping help ensure $(\epsilon, \delta)$-differential privacy,
making it ideal for sensitive applications.
Previous work shows that DP-SGD, when optimized, can yield near state-of-the-art accuracy
for differentially private models~\cite{De_2022_ScaleDP_ImageClassification}.

\subsection{Why This Approach Will Outperform Others}\label{subsec:why-this-approach-will-outperform-others}
Our approach leverages the ResNet-20 model which is often used in conjunction with
CIFAR-10 dataset.
Similarly, DP-SGD is an often used optimization algorithm, that has proven powerful in balancing the privacy-utility trade-offs posed
by differential privacy, as seen in \cite{Abadi_2016_DeepLearningDifferentialPrivacy}. With the baseline
of DP-SGD and ResNet-20, we believe our incremental improvements will yield strong gains in
accuracy.
Converting from DP-SGD to DP-RMSprop may improve the accuracy by adapting the learning rate as the
model trains.
This is especially important given the limited amount of times we are able to
query a dataset while implementing differential privacy.
Additionally, we believe the added gradient normalization introduced by upgrading to DP-ADAM will
similarly improve the rate of convergence of our model as seen in the non-private study \cite{DBLP:journals/corr/abs-1807-06766}. Again, this is paramount given the
limited number of times that our model can query the dataset.
Finally, by experimenting with automatic clipping, we aim to
find an optimal trade-off between accuracy and privacy.

\subsection{Pseudocode for Non-Private Optimizers}\label{subsec:pseudo-code-for-non-private-optimizers}
Below are pseudocodes for the non-private SGD, non-private RMSprop, and non-private ADAM algorithms that we plan to privatize, adapted or referenced from \cite{DBLP:journals/corr/abs-1807-06766}:
\begin{algorithm}
    \caption{SGD Algorithm}
    \label{alg:sgd}
    \begin{algorithmic}[1]
        \State \textbf{Input:} A step size $\alpha$, initial starting point $\mathbf{x}_1 \in \mathbb{R}^d$,
        and access to a (possibly noisy) oracle for gradients of $f : \mathbb{R}^d \rightarrow \mathbb{R}$.
        \Function{SGD}{$\mathbf{x}_1, \alpha$}
            \State Initialize: $\mathbf{v}_0 = \mathbf{0}$
            \For{$t = 1, 2, \dots$}
                \State $\mathbf{g}_t = \nabla f(\mathbf{x}_t)$
                \State $\mathbf{x}_{t+1} = \mathbf{x}_t - \alpha \mathbf{g}_t$
            \EndFor
        \EndFunction
    \end{algorithmic}
\end{algorithm}
\vspace{-1cm}

\begin{algorithm}
    \caption{RMSProp}
    \label{alg:rmsprop}
    \begin{algorithmic}[1]
        \State \textbf{Input:} A constant vector $\mathbb{R}^d \ni \xi \mathbf{1}_d \geq 0$, parameter $\beta_2 \in [0, 1)$, step size $\alpha$, initial starting point $\mathbf{x}_1 \in \mathbb{R}^d$, and access to a (possibly noisy) oracle for gradients of $f : \mathbb{R}^d \rightarrow \mathbb{R}$.
        \Function{RMSProp}{$\mathbf{x}_1, \beta_2, \alpha, \xi$}
            \State Initialize: $\mathbf{v}_0 = \mathbf{0}$
            \For{$t = 1, 2, \dots$}
                \State $\mathbf{g}_t = \nabla f(\mathbf{x}_t)$
                \State $\mathbf{v}_t = \beta_2 \mathbf{v}_{t-1} + (1 - \beta_2)(\mathbf{g}_t^2 + \xi \mathbf{1}_d)$
                \State $\mathbf{V}_t = \text{diag}(\mathbf{v}_t)$
                \State $\mathbf{x}_{t+1} = \mathbf{x}_t - \alpha \mathbf{V}_t^{-\frac{1}{2}} \mathbf{g}_t$
            \EndFor
        \EndFunction
    \end{algorithmic}
\end{algorithm}
\vspace{-1cm}

\begin{algorithm}
    \caption{ADAM}
    \label{alg:adam}
    \begin{algorithmic}[1]
        \State \textbf{Input:} A constant vector $\mathbb{R}^d \ni \xi \mathbf{1}_d > 0$, parameters $\beta_1, \beta_2 \in [0, 1)$, a sequence of step sizes $\{\alpha_t\}_{t=1,2,\dots}$, initial starting point $\mathbf{x}_1 \in \mathbb{R}^d$, and oracle access to the gradients of $f : \mathbb{R}^d \to \mathbb{R}$.
        \Function{ADAM}{$\mathbf{x}_1, \beta_1, \beta_2, \{\alpha_t\}, \xi$}
            \State Initialize: $\mathbf{m}_0 = \mathbf{0}$, $\mathbf{v}_0 = \mathbf{0}$
            \For{$t = 1, 2, \dots$}
                \State $\mathbf{g}_t = \nabla f(\mathbf{x}_t)$
                \State $\mathbf{m}_t = \beta_1 \mathbf{m}_{t-1} + (1 - \beta_1) \mathbf{g}_t$
                \State $\mathbf{v}_t = \beta_2 \mathbf{v}_{t-1} + (1 - \beta_2) \mathbf{g}_t^2$
                \State $\mathbf{V}_t = \operatorname{diag}(\mathbf{v}_t)$
                \State $\mathbf{x}_{t+1} = \mathbf{x}_t - \alpha_t \left( \mathbf{V}_t^{\frac{1}{2}} + \operatorname{diag}(\xi \mathbf{1}_d) \right)^{-1} \mathbf{m}_t$
            \EndFor
        \EndFunction
    \end{algorithmic}
\end{algorithm}

    \break
    \section{Experimental Setup}\label{sec:experimental-setup}
    \subsection{System Description}\label{subsec:system-description}
We will use PyTorch~\cite{pytorch_2019} for model implementation and training.
The DP-SGD~\cite{Abadi_2016_DeepLearningDifferentialPrivacy} implementation will be based on the Opacus library~\cite{opacus}.
Training will be performed on GPUs available via our departmental server csci-cuda.cs.uga.edu or on Google Colab.

\subsection{Dataset}\label{subsec:dataset}
We will use the CIFAR-10 dataset~\cite{cifar10_dataset}, consisting of 60,000 32x32 RGB images, which is commonly used for
image classification tasks.
The dataset is built-in in PyTorch~\cite{pytorch_2019}, and we will load it using standard libraries.

\subsection{Metrics to Compare}\label{subsec:metrics}
\begin{itemize}
    \item \textbf{Accuracy/Training Loss:} Accuracy and training loss on CIFAR-10~\cite{cifar10_dataset}.
    \item \textbf{Privacy Cost:} We will measure $(\epsilon, \delta)$ using RDP~\cite{Mironov_2017_RenyiDP} to ensure privacy compliance.
    \item \textbf{Training Time:} Time complexity and memory usage will be tracked.
\end{itemize}

\subsection{Design of Experiments}\label{subsec:design-of-experiments}
We will be performing a series of experiments to evaluate our modified differentially private optimizers against other baseline models.
Table~\ref{tab:doe} outlines the experimental design, including the optimizer, clipping method, noise mechanism, and other metrics.

\begin{table}[!ht]
    \caption{Experimental Design}  % Title of the table
    \centering  % Center the table
    \resizebox{\textwidth}{!}{  % Resize the table to fit the width of the page
        \begin{tabular}{|c|c|c|c|c|c|c|c|}
            \hline
            \textbf{Experiment ID} & \textbf{Optimizer} & \textbf{Clipping Method} & \textbf{Noise Mechanism}   & \textbf{Accuracy} & \textbf{Training Time} & \textbf{Privacy Cost}\\ [0.5ex]
            \hline\hline
            1                      & DP-SGD             & Standard                 & Standard Gaussian        & TBD               & TBD                    & TBD                   \\
            2                      & DP-RMSprop         & Standard                 & Standard Gaussian        & TBD               & TBD                    & TBD                   \\
            3                      & DP-Adam            & Standard                 & Standard Gaussian        & TBD               & TBD                    & TBD                   \\
            4                      & DP-SGD             & Automatic Clipping       & Standard Gaussian        & TBD               & TBD                    & TBD                   \\
            5                      & DP-RMSprop         & Automatic Clipping       & Standard Gaussian        & TBD               & TBD                    & TBD                   \\
            6                      & DP-Adam            & Automatic Clipping       & Standard Gaussian        & TBD               & TBD                    & TBD                   \\
           \hline
        \end{tabular}
    } % End of \resizebox
    \label{tab:doe}  % Label of the table
\end{table}

\subsection{Baseline Models}\label{subsec:baseline-models}
We will compare the performance of our modified differentially private models with standard private optimizers,
including vanilla DP-SGD~\cite{Abadi_2016_DeepLearningDifferentialPrivacy}, DP-RMSprop,
and DP-Adam~\cite{zhou_2020_private_adaptive_algorithms}.
Additionally, we will benchmark against AdaClip~\cite{adaClip_2019} to evaluate the effectiveness of our automatic
clipping and noise mechanism modifications.

    \section{Related Work}\label{sec:related-work}
    Several approaches to differentially private deep learning have been explored in the literature.
    Abadi et al. \cite{Abadi_2016_DeepLearningDifferentialPrivacy} introduced DP-SGD, which has become a foundational technique for
    privacy-preserving model training.
    Our focus is on making incremental improvements to this framework by adapting it to DP-RMSprop and DP-Adam.
    Additionally, we aim to incorporate automatic clipping~\cite{bu2023automaticclippingdifferentiallyprivate}.


    \section{Timeline and Milestones}\label{sec:timeline-and-milestones}
    \begin{itemize}
        \item October 4, 2024: Best Accuracy Report 1 Due.
        \item October 18, 2024: Interim Report Due.
        \item November 8, 2024: Best Accuracy Report 2 Due.
        \item November 22, 2024: Final Report Due.
    \end{itemize}

    \section{Best Accuracy Report \#1}\label{sec:best-accuracy-report}
    \subsection{Current Implementation Overview}\label{subsec:current-implementation}
\begin{table}[!ht]

    \caption{Experimental Results}  % Title of the table

    \centering  % Center the table

    \resizebox{\textwidth}{!}{  % Resize the table to fit the width of the page

        \begin{tabular}{|c|c|c|c|c|c|c|c|}

            \hline

            \textbf{Experiment ID} & \textbf{Optimizer} & \textbf{Epochs} & \textbf{Accuracy} & \textbf{Training Time (s)} & \textbf{Privacy Cost} & \textbf{Learning Rate} & \textbf{Batch Size} \\ [0.5ex]

            \hline\hline

            1  & SGD    & 100 & 87\% & -     & -     & 0.1 & 128  \\

            2  & SGD    & 200 & 94\% & -     & -     & 0.1 & -    \\

            3  & DP-SGD & 100 & 39\% & -     & -     & 0.1 & 128  \\

            4  & DP-SGD & 30  & 40\% & 598.68 & 3    & 0.2 & 128  \\

            5  & DP-SGD & 30  & 39\% & 527.37 & 3    & 0.3 & 128  \\

            6  & DP-SGD & 30  & 37\% & 584.55 & 3    & 0.4 & 128  \\

            7  & DP-SGD & 30  & 38\% & 597.60 & 3    & 0.5 & 128  \\

            9  & DP-SGD & 30  & 35\% & 995.68 & 3    & 0.1 & 64   \\

            10 & DP-SGD & 30  & 44\% & 473.86 & 3    & 0.1 & 256  \\

            11 & DP-SGD & 30  & 44\% & 597.29 & 2.99 & 0.1 & 512  \\

            12 & DP-SGD & 30  & 42\% & 677.04 & 3    & 0.1 & 1024 \\

            13 & DP-SGD & 30  & 43\% & 519.55 & 8.01 & 0.1 & 128  \\

            14 & DP-SGD & 30  & 44\% & 627.49 & 10.01 & 0.1 & 128  \\

            15 & DP-SGD & 30  & 48\% & 553.12 & 50.04 & 0.1 & 128  \\

           \hline

        \end{tabular}

    } % End of \resizebox

    \label{tab:exp_results}  % Label of the table

\end{table}

 




\subsection{Best Observed Accuracy and Component that Attributed}\label{subsec:best-accuracy}

\subsection{Hyperparamters used}\label{subsec:hyperparameters}

\subsection{Failed Approaches}\label{subsec:failed-approaches}

\subsection{Training Methods}\label{subsec:training-methods}

    \section{Best Accuracy Report \#2}\label{sec:best-accuracy-report2}
    \subsection{Current Implementation Overview}\label{subsec:current-implementation}

\begin{table}[!ht]
    \caption{Experimental Results\\All experiments were conducted with a constant privacy budget $\delta = 10^{-5}$, momentum $\beta = 0.9$, weight decay    $\lambda = 10^{-4}$ and a maximum gradient norm of $C = 1.0$.}
    \centering  % Center the table
    \resizebox{\textwidth}{!}{  % Resize the table to fit the width of the page
        \begin{tabular}{|c|c|c|c|c|c|c|c|c|}
            \hline

            \textbf{Experiment ID} & \textbf{Optimizer} & \textbf{Epochs} & \textbf{Accuracy} & \textbf{Training Time (s)} & \textbf{Privacy Cost} & \textbf{Learning Rate} & \textbf{Batch Size} & \textbf{Noise Multiplier} \\ [0.5ex]
            \hline\hline
            1                      & SGD                & 100             & 87\%              & -                          & -                     & 0.1                    & 128                 & -                         \\
            2                      & SGD                & 200             & 94\%              & -                          & -                     & 0.1                    & 128                 & -                         \\
            \textbf{3}             & \textbf{DP-SGD}    & \textbf{30}     & \textbf{41\%}     & \textbf{481.49}            & \textbf{3}            & \textbf{0.1}           & \textbf{128}        & \textbf{1.1}              \\
            4                      & DP-SGD             & 30              & 40\%              & 598.68                     & 3                     & 0.2                    & 128                 & 1.1                       \\
            5                      & DP-SGD             & 30              & 39\%              & 527.37                     & 3                     & 0.3                    & 128                 & 1.1                       \\
            6                      & DP-SGD             & 30              & 37\%              & 584.55                     & 3                     & 0.4                    & 128                 & 1.1                       \\
            7                      & DP-SGD             & 30              & 38\%              & 597.60                     & 3                     & 0.5                    & 128                 & 1.1                       \\
            8                      & DP-SGD             & 30              & 35\%              & 995.68                     & 3                     & 0.1                    & 64                  & 1.1                       \\
            \textbf{9}             & \textbf{DP-SGD}    & \textbf{30}     & \textbf{44\%}     & \textbf{473.86}            & \textbf{3}            & \textbf{0.1}           & \textbf{256}        & \textbf{1.1}              \\
            10                     & DP-SGD             & 30              & 44\%              & 597.29                     & 2.99                  & 0.1                    & 512                 & 1.1                       \\
            11                     & DP-SGD             & 30              & 42\%              & 677.04                     & 3                     & 0.1                    & 1024                & 1.1                       \\
            12                     & DP-SGD             & 30              & 43\%              & 519.55                     & 8.01                  & 0.1                    & 128                 & 1.1                       \\
            13                     & DP-SGD             & 30              & 44\%              & 627.49                     & 10.01                 & 0.1                    & 128                 & 1.1                       \\
            14                     & DP-SGD             & 30              & 48\%              & 553.12                     & 50.04                 & 0.1                    & 128                 & 0.1                       \\
            \textbf{15}            & \textbf{DP-SGD}    & \textbf{30}     & \textbf{50\%}     & \textbf{375.37}            & \textbf{50.04}        & \textbf{0.1}           & \textbf{256}        & \textbf{0.1}              \\

            \hline
        \end{tabular}
    } % End of \resizebox
    \label{tab:exp_results}  % Label of the table
\end{table}

\subsection{Best Observed Accuracy and Components/Hyperparameters}\label{subsec:best-accuracy}
In the Experimental testing above, we varied hyperparameters to maximize accuracy of classification on the CIFAR10 dataset. We employed the
($\epsilon$, $\delta$)-Differentially Private - Stochastic Gradient Descent (DP-SGD) as our optimizer. For our loss function, we used the built-in PyTorch CrossEntropy function.
The bolded lines, in the table above, indicate the maximum accuracies achieved by varying each of the hyperparameters. As a baseline, we also implemented
a non-private SGD to give an upper bound on potential accuracy for the CIFAR10 dataset. We were able to achieve 94\% accuracy with non-private SGD over 200 epochs.
\begin{itemize}
    \item \textbf{Learning Rate:} We varied learning rate from 0.1 to 0.5. The smallest learning rate (0.1) yielded the highest accuracy of 41\%.
    \item \textbf{Batch Size:} We varied batch size from 64 to 1024. A batch size of 256 yielded the highest accuracy of 44\% among its peers.
    \item \textbf{Epsilon/Privacy Budget:} Higher epsilons mean lower privacy, but also can mean higher utility. When the epsilon was adjusted to 50, accuracy peaked at 50\%.
    \item \textbf{Noise Multiplier:} Along with Epsilon changes, our highest accuracy (50\%) run also coincided with a decrease in noise multiplier (0.1).
\end{itemize}

\subsection{Failed Approaches}\label{subsec:failed-approaches}
Below is discussion of the approaches that either failed to improve accuracy or reduced the accuracy from our first DP-SGD model run (Experiment ID \#3).

\begin{itemize}
    \item \textbf{Increasing learning rate:} Adjusting the learning rate from 0.3 to 0.5 still yielded lower accuracy, ranging between 37\% and 39\%.
    \item \textbf{Adjusting Batch Sizes:} Decreasing the batch size to 64 was a failure on 2 fronts. It first decreased the accuracy of the model to 35\% and it was noticeably slower than all other approaches at more than 900 seconds to completion. Higher Batch Sizes, near 1000, also experienced some slowdown (more than 650 seconds). Any batch size other than the 256 or 512 seemed to have no positive effect on accuracy. CUDA also indicated a memory warning when batch size was 1024.
\end{itemize}

\subsection{Implementation Challenges}\label{subsec:implementation-challenges}
In the course of completing these tests, we ran into numerous issues. These issues mainly during the implementation of the Opacus Privacy Engine. Our original implementation builds a Residual Network
with 20 layers (Resnet20) and uses non-private SGD as its optimizer.
\begin{itemize}
    \item \textbf{BatchNorm to GroupNorm:} Opacus was not compatible with BatchNorm (which was used in our base implementation \cite{Idelbayev_ResNet20}). To fix this, we employed Opacus' ModuleValidator.fix(model) built-in function, which replaces all the BatchNorms with GroupNorms.
    \item \textbf{Lambda Layer:} Next, we replaced the Lambda layer (from the base implementation \cite{Idelbayev_ResNet20}) with a Shortcut layer, as the Lambda layer was using Serializable functions which weren't compatible with ModuleValidator.fix(model). This occurs because Lambda layers use unnamed functions which are not serializable.
    \item \textbf{Dead Module:} Lastly, we encountered an error with a "dead module". Specifically, while calling the loss.backward() function, our model would return an RuntimeError indicating that we were trying to call the hook of a dead module. This arose due to the base implementations \cite{Idelbayev_ResNet20} usage of torch.nn.DataParallel. To fix this, we removed this parallelization from the code.

\end{itemize}



    \bibliographystyle{plain}
    \bibliography{references}

    \section*{GitHub Contributions}
    The code and related materials for this project are available at our GitHub repository:
    \url{https://github.com/CS8960-Privacy-Preserving-Data-Analysis/final-project}.
    Contributions, issues, and discussions are welcome.

    \appendix
    \begin{appendices}  % If using the 'appendix' package
    \section{Detailed Experiment Results}\label{sec:detailed-experiment-results}\nopagebreak
    
\begin{landscape}

    \renewcommand{\arraystretch}{1.3}
    \begin{table}[ht]
        \centering
        \normalsize
        \begin{adjustbox}{max width=\linewidth, max height=\textheight}
            \begin{tabular}{@{}lllcc*{5}{cc}ccccc@{}}
                \toprule
                \textbf{Exp ID} & \textbf{Opt} & \textbf{Epochs} & \textbf{Avg Acc} &
                \multicolumn{2}{c}{\textbf{Run 1}} &
                \multicolumn{2}{c}{\textbf{Run 2}} &
                \multicolumn{2}{c}{\textbf{Run 3}} &
                \multicolumn{2}{c}{\textbf{Run 4}} &
                \multicolumn{2}{c}{\textbf{Run 5}} &
                \textbf{Privacy Cost} & \rotatebox{90}{\textbf{LR}} & \rotatebox{90}{\textbf{Batch}} & \rotatebox{90}{\textbf{Noise}} & \rotatebox{90}{\textbf{Beta 1}} & \rotatebox{90}{\textbf{Beta 2}} \\
                \cmidrule(lr){5-6}
                \cmidrule(lr){7-8}
                \cmidrule(lr){9-10}
                \cmidrule(lr){11-12}
                \cmidrule(lr){13-14}
                & & & &
                \textbf{Acc} & \textbf{Time} &
                \textbf{Acc} & \textbf{Time} &
                \textbf{Acc} & \textbf{Time} &
                \textbf{Acc} & \textbf{Time} &
                \textbf{Acc} & \textbf{Time} &
                & & & & & \\
                % Data Rows Start Here
                \midrule
                1 & DP-Lion & 30 & 42.01\% &
                43.35\% & 489.8 &
                43.67\% & 690.09 &
                40.29\% & 677.16 &
                41.62\% & 678.1 &
                41.10\% & 677.23 &
                7.99 & 0.003 & 256 & 1.1 & 0.9 & 0.999 \\
                \midrule
                2 & DP-Lion & 30 & 41.41\% &
                41.11\% & 1093.4 &
                41.95\% & 1088.26 &
                41.99\% & 1110.42 &
                42.00\% & 1116.9 &
                41.72\% & 1139.94 &
                7.99 & 0.0003 & 64 & 1.1 & 0.9 & 0.999 \\
                \midrule
                3 & DP-Lion & 30 & 37.13\% &
                42.74\% & 707.22 &
                34.50\% & 695.22 &
                35.79\% & 664.77 &
                36.90\% & 559.75 &
                35.72\% & 566.71 &
                7.99 & 0.0003 & 128 & 1.1 & 0.9 & 0.999 \\
                \midrule
                4 & DP-Lion & 30 & 43.90\% &
                43.78\% & 765.43 &
                44.46\% & 498.97 &
                41.37\% & 500.11 &
                45.79\% & 580.04 &
                44.08\% & 642.37 &
                7.99 & 0.0003 & 512 & 1.1 & 0.9 & 0.999 \\
                \midrule
                5 & DP-Lion & 30 & 41.72\% &
                41.08\% & 953.57 &
                41.80\% & 821.57 &
                43.85\% & 774.07 &
                39.77\% & 763.44 &
                42.12\% & 541.74 &
                7.99 & 0.0003 & 1024 & 1.1 & 0.9 & 0.999 \\
                \midrule
                6 & DP-Lion & 30 & 37.71\% &
                37.49\% & 1364.81 &
                38.20\% & 718.85 &
                34.78\% & 688.51 &
                40.08\% & 840.74 &
                38.02\% & 694.35 &
                7.99 & 0.0003 & 2048 & 1.1 & 0.9 & 0.999 \\
                \midrule
                7 & DP-Lion & 30 & 32.36\% &
                32.72\% & 687.02 &
                33.79\% & 387.59 &
                34.33\% & 387.38 &
                29.98\% & 429.58 &
                30.96\% & 618.34 &
                7.99 & 0.1 & 256 & 1.1 & 0.9 & 0.999 \\
                \midrule
                8 & DP-Lion & 30 & 40.99\% &
                40.52\% & 693.77 &
                40.62\% & 621.77 &
                42.58\% & 404.77 &
                40.97\% & 625.35 &
                40.26\% & 390.1 &
                7.99 & 0.01 & 256 & 1.1 & 0.9 & 0.999 \\
                \midrule
                9 & DP-Lion & 30 & 44.92\% &
                45.21\% & 697.67 &
                45.13\% & 489.4 &
                45.09\% & 491.66 &
                44.24\% & 488.25 &
                44.91\% & 603.59 &
                7.99 & 0.001 & 256 & 1.1 & 0.9 & 0.999 \\
                \midrule
                10 & DP-Lion & 30 & 40.88\% &
                40.66\% & 688.58 &
                41.77\% & 849.46 &
                40.52\% & 500.54 &
                40.76\% & 489.32 &
                40.69\% & 493.58 &
                7.99 & 0.0001 & 256 & 1.1 & 0.9 & 0.999 \\
                \midrule
                11 & DP-Lion & 30 & 44.66\% &
                45.50\% & 1353.32 &
                44.83\% & 713.72 &
                43.07\% & 684.61 &
                45.33\% & 700.32 &
                44.55\% & 689.72 &
                7.99 & 0.001 & 2048 & 2.1 & 0.9 & 0.999 \\
                \midrule
                12 & DP-Lion & 30 & 44.52\% &
                45.16\% & 737.86 &
                44.36\% & 692.99 &
                45.23\% & 698.26 &
                42.80\% & 681.08 &
                45.07\% & 720.68 &
                7.99 & 0.001 & 2048 & 3.1 & 0.9 & 0.999 \\
                \midrule
                13 & DP-Lion & 30 & 45.55\% &
                46.10\% & 815.71 &
                45.22\% & 910.33 &
                45.59\% & 715.69 &
                45.45\% & 774.95 &
                45.38\% & 936.26 &
                7.99 & 0.001 & 2048 & 4.1 & 0.9 & 0.999 \\
                \midrule
                14 & DP-Lion & 30 & 44.72\% &
                45.30\% & 733.84 &
                45.17\% & 704.37 &
                44.78\% & 960.33 &
                43.24\% & 698.53 &
                45.11\% & 926.9 &
                7.99 & 0.001 & 2048 & 5.1 & 0.9 & 0.999 \\
                \midrule
                15 & DP-Lion & 30 & 26.72\% &
                45.04\% & 766.2 &
                45.23\% & 722.57 &
                43.33\% & 680.96 &
                -- & -- &
                -- & -- &
                7.99 & 0.001 & 2048 & 6.1 & 0.9 & 0.999 \\
                \midrule
                16 & DP-Lion & 30 & 26.32\% &
                45.22\% & 1044.46 &
                42.86\% & 869.19 &
                43.52\% & 827.31 &
                -- & -- &
                -- & -- &
                7.99 & 0.001 & 2048 & 10.1 & 0.9 & 0.999 \\
                \midrule
                17 & DP-Lion & 30 & 42.40\% &
                42.55\% & 574.75 &
                41.53\% & 497.16 &
                43.05\% & 487.76 &
                42.10\% & 486.32 &
                42.77\% & 480.27 &
                3 & 0.001 & 256 & 1.1 & 0.9 & 0.999 \\
                \midrule
                18 & DP-Lion & 30 & 45.76\% &
                45.53\% & 571.52 &
                45.77\% & 353.65 &
                45.66\% & 569.54 &
                45.63\% & 516.23 &
                46.20\% & 535.55 &
                10 & 0.001 & 256 & 1.1 & 0.9 & 0.999 \\
                \midrule
                19 & DP-Lion & 30 & 50.59\% &
                51.65\% & 536.85 &
                50.31\% & 597.86 &
                49.86\% & 428.46 &
                50.37\% & 423.37 &
                50.76\% & 421.25 &
                50 & 0.001 & 256 & 1.1 & 0.9 & 0.999 \\
                \midrule
                20 & DP-Lion & 200 & 10.37\% &
                51.86\% & 3549.89 &
                -- & -- &
                -- & -- &
                -- & -- &
                -- & -- &
                7.99 & 0.001 & 2048 & 10.1 & 0.9 & 0.999 \\
                \midrule
                21 & DP-Lion & 200 & 10.32\% &
                51.59\% & 3514.54 &
                -- & -- &
                -- & -- &
                -- & -- &
                -- & -- &
                7.99 & 0.001 & 2048 & 4.1 & 0.9 & 0.999 \\
                \midrule
                22 & DP-Lion & 200 & 32.55\% &
                54.43\% & 3544.98 &
                54.09\% & 3848.46 &
                54.24\% & 3788.61 &
                -- & -- &
                -- & -- &
                7.99 & 0.001 & 2048 & 10.1 & 0.95 & 0.98 \\
                % Data Rows End Here
                \bottomrule
            \end{tabular}
        \end{adjustbox}
        \caption{Detailed Results of All Experiment Runs}\label{tab:detailed-results}
    \end{table}
\end{landscape}

    \end{appendices}


\end{document}
