\documentclass{article}

\usepackage[preprint]{neurips_2024}
\usepackage[utf8]{inputenc}
\usepackage[T1]{fontenc}
\usepackage{hyperref}
\usepackage{url}
\usepackage{booktabs}
\usepackage{amsfonts}
\usepackage{nicefrac}
\usepackage{microtype}
\usepackage{xcolor}
\usepackage{graphicx}
\usepackage{algorithm}
\usepackage{algpseudocode}
\usepackage{amsmath}
\usepackage{graphicx}
\usepackage{array}

\title{Differential Privacy in Image Classification using ResNet-20 and DP-SGD Optimization}

\author{
    Praveen Rangavajhula\\
    Department of Computer Science\\
    University of Georgia\\
    Athens, GA, 30602\\
    \texttt{praveen.rangavajhula@uga.edu} \\
    \And
    Alexander Darwiche\\
    Department of Computer Science\\
    University of Georgia\\
    Athens, GA, 30605 \\
    \texttt{alexander.darwiche@uga.edu} \\
    \And
    Deven Allen\\
    Department of Computer Science\\
    University of Georgia\\
    Athens, GA, 30605 \\
    \texttt{dca09692@uga.edu} \\
}

\begin{document}

    \maketitle

    \begin{abstract}

        This project proposes a differentially private image classification system using ResNet-20 with various
        optimizers, starting with Differentially Private Stochastic Gradient Descent (DP-SGD) as a baseline.
        We aim to make incremental improvements with additional optimization techniques, exploring both non-private
        and DP versions of optimizers, and justifying our choices based on prior work and their potential
        to outperform others.
        The project will focus on achieving competitive accuracy while satisfying privacy guarantees.
        Additionally, we are investigating ways to enhance accuracy by modifying optimizer components,
        such as gradient clipping (potentially using techniques like automatic clipping).

    \end{abstract}


    \section{Introduction}\label{sec:introduction}

    The increase in prevalence of machine learning models, especially in image classification,
    has coincided with concerns over privacy~\cite{papernot2022hyperparametertuningrenyidifferential}.
    Differential privacy (DP) specifically addresses these concerns by ensuring that models do not inadvertently leak sensitive
    information about individual data points.
    In this proposal, we will use the ResNet-20 model, which is well-suited for datasets like CIFAR-10~\cite{Idelbayev_ResNet20},
    and will implement and improve DP-SGD to achieve better privacy guarantees without significantly compromising accuracy.


    \section{Motivation and Problem Statement}\label{sec:motivation-and-problem-statement}
    In many real-world applications, the concerns over privacy leakage can hinder the deployment of machine
    learning models.
    Current state-of-the-art models like ResNet-20 achieve high accuracy but are vulnerable to attacks that could leak
    sensitive information.
    An example of a differentially private optimizer is Differentially Private Stochastic Gradient Descent (DP-SGD), which has been shown
    to effectively reduce privacy leaks, but has challenges in balancing model accuracy and privacy~\cite{Abadi_2016_DeepLearningDifferentialPrivacy}.
    Even with the strong foundation that DP-SGD provides, we believe there is room for improvements that can
    achieve both higher accuracy and stronger privacy guarantees.



    \section{Methodology}\label{sec:methodology}
    \subsection{Model Architecture: ResNet-20}\label{subsec:model-architecture:-resnet-20}


We propose utilizing the ResNet-20 model~\cite{Idelbayev_ResNet20} for CIFAR-10,
a standard dataset for image classification tasks.
We selected a 20-layer ResNet for its deep architecture and strength in image classification problems~\cite{DBLP:journals/corr/HeZRS15}.
20
layers should be enough depth to adequately model many features, while not encountering the higher training error encountered
with excessively \("\)deep\("\) architectures. \cite{DBLP:journals/corr/HeZRS15}

If necessary, we may modify the architecture slightly to optimize for DP compatibility.

\subsection{Non-private Optimizers to Try}\label{subsec:non-private-optimizers-to-try}
We propose implementing 4 non-private optimizers to establish baseline performance for ResNet-20 on CIFAR-10.
Three of these optimizers will be First-Order optimizers
that all build on one another.
The last optimizer that we will explore is a Second-Order optimization technique known as Cubic .

\begin{itemize}
    \item \textbf{SGD:} Standard Stochastic Gradient Descent for baseline comparison.
    This optimizer works by calculating the gradient at each data
    point and updating the model parameters with the following update rule: $\theta + \eta * \Delta g$, where $\theta$ is model parameters, $\eta$ is
    the learning rate, and $\Delta g$ is the gradient at that data point.
    \item \textbf{RMSprop:} Root Mean Square Propagation (RMSprop) builds on SGD by including the \("\)moving average\("\) factor.
    This factor functions by scaling the gradient
    each step, based on the gradient of the previous data points.
    This is done by dividing the gradient, at each model parameter update,
    by the moving average squared gradient: $(\Delta g_{t-1}^{2} * (1-\delta) + \Delta g{t}^{2}*\delta)^{\frac{1}{2}}$,
    where $\Delta g_{t-1}^{2}$ is the squared gradient average from the previous step, $\Delta g_{t}^{2}$ is
    the squared gradient of the current step, and $\delta$ is the \("\)moving average\("\) factor.  \cite{DBLP:journals/corr/abs-1807-06766,Jason_Huang_2020}
    \item \textbf{ADAM:} Adaptive Moment Estimation (ADAM) further build on RMSprop by including another \("\)moving average\("\) factor, this time for the gradient.
    In the general gradient update rule formula, instead of gradient,
    ADAM substitutes in the gradient moving average: $m_{t} = \Delta g_{t-1} * (1-\delta) + \Delta g{t}*\delta$~\cite{DBLP:journals/corr/abs-1807-06766}
\end{itemize}

\subsection{Differentially Private Optimizer: DP-SGD}\label{subsec:differentially-private-optimizer:-dp-sgd}
We propose implementing DP-SGD as our privacy-preserving algorithm.
The key components of DP-SGD are:
\begin{itemize}
    \item \textbf{Gradient Clipping:} Limits the influence of individual examples during training.
    \item \textbf{Noise Addition:} Adds noise to gradients to ensure privacy (via Opacus or TensorFlow Privacy libraries).
    \item \textbf{Privacy Accounting:} We will use Rényi Differential Privacy (RDP) for privacy budget tracking.
\end{itemize}

\subsection{Incremental Improvements}\label{subsec:incremental-improvements}
After we have privatized SGD, we propose making the following enhancements:
\begin{itemize}
    \item Adding an adaptive learning rate by incorporating the Moving Average of Gradients Squared (RMSprop)~\cite{DBLP:journals/corr/abs-1807-06766}
    \item Adding a moving average for gradient (ADAM)~\cite{DBLP:journals/corr/abs-1807-06766}
    \item Alternative noise mechanisms and their effect on utility-privacy tradeoffs. [NEED JUSTIFICATION]
    \item Adaptive gradient clipping methods that dynamically adjust the clipping threshold. [NEED JUSTIFICATION]
\end{itemize}

\subsection{Rationale for Choosing DP-SGD}\label{subsec:rationale-for-choosing-dp-sgd}
\begin{itemize}
    \item provides well-documented privacy guarantees~\cite{Abadi_2016_DeepLearningDifferentialPrivacy}
    while maintaining decent utility for image classification tasks.
    \item The addition of noise and gradient clipping help ensure $(\epsilon, \delta)$-differential privacy,
    making it ideal for sensitive applications.
    \item Previous work shows that DP-SGD, when optimized, can yield near state-of-the-art accuracy
    for differentially private models~\cite{De_2022_ScaleDP_ImageClassification}.
\end{itemize}

\subsection{Why This Approach Will Outperform Others}\label{subsec:why-this-approach-will-outperform-others}
\begin{itemize}
    \item Our approach leverages the simplicity of ResNet-20, optimized for smaller datasets like CIFAR-10,
    combined with DP-SGD, a proven differential privacy technique.
    \item By experimenting with different gradient clipping techniques and noise scales,
    we aim to find an optimal trade-off between accuracy and privacy.
    \item We will investigate modifications like adaptive clipping to enhance performance.
\end{itemize}

\subsection{Pseudocode for Non-Private SGD}\label{subsec:pseudo-code-for-non-private-sgd}
Below is a simplified pseudocode for the non-private SGD we plan to privatize:
\begin{verbatim}
    for each batch (X, y):
        pred = model(X)
        loss = loss_fn(pred, y)
        loss.backward()
        optimizer.step()
        optimizer.zero_grad()
\end{verbatim}

    \break
    \section{Experimental Setup}\label{sec:experimental-setup}
    \subsection{System Description}\label{subsec:system-description}
We will use PyTorch~\cite{pytorch_2019} for model implementation and training.
The DP-SGD~\cite{Abadi_2016_DeepLearningDifferentialPrivacy} implementation will be based on the Opacus library~\cite{opacus}.
Training will be performed on GPUs available via our departmental server csci-cuda.cs.uga.edu.

\subsection{Dataset}\label{subsec:dataset}
We will use the CIFAR-10 dataset~\cite{cifar10_dataset}, consisting of 60,000 32x32 RGB images, which is commonly used for
image classification tasks.
The dataset is built-in in PyTorch~\cite{pytorch_2019}, and we will load it using standard libraries.

\subsection{Metrics}\label{subsec:metrics}
\begin{itemize}
    \item \textbf{Training Loss/Accuracy:} Standard accuracy and training loss on CIFAR-10~\cite{cifar10_dataset}.
    \item \textbf{Privacy Budget:} We will measure $(\epsilon, \delta)$ using RDP~\cite{Mironov_2017_RenyiDP} to ensure privacy compliance.
    \item \textbf{Efficiency:} Time complexity and memory usage will be tracked.
\end{itemize}

\subsection{Design of Experiments}\label{subsec:design-of-experiments}
We will be performing a series of experiments to evaluate our modified differentially private optimizers against other baseline models.
Table~\ref{tab:doe} outlines the experimental design, including the optimizer, clipping method, noise mechanism, and other metrics.

\begin{table}[!ht]
    \centering  % Center the table
    \resizebox{\textwidth}{!}{  % Resize the table to fit the width of the page
        \begin{tabular}{||c|c|c|c|c|c|c|c||}
            \hline
            \textbf{Experiment ID} & \textbf{Optimizer} & \textbf{Clipping Method} & \textbf{Noise Mechanism}   & \textbf{Accuracy} & \textbf{Training Time} & \textbf{Privacy Cost}\\ [0.5ex]
            \hline\hline
            1                      & DP-SGD             & Standard                 & Standard Gaussian        & TBD               & TBD                    & TBD                   \\
            2                      & DP-RMSprop         & Standard                 & Standard Gaussian        & TBD               & TBD                    & TBD                   \\
            3                      & DP-Adam            & Standard                 & Standard Gaussian        & TBD               & TBD                    & TBD                   \\
            4                      & DP-SGD             & Automatic Clipping       & Standard Gaussian        & TBD               & TBD                    & TBD                   \\
            5                      & DP-RMSprop         & Automatic Clipping       & Standard Gaussian        & TBD               & TBD                    & TBD                   \\
            6                      & DP-Adam            & Automatic Clipping       & Standard Gaussian        & TBD               & TBD                    & TBD                   \\
            7                      & DP-SGD             & Adaptive Clipping        & Standard Gaussian        & TBD               & TBD                    & TBD                   \\
            8                      & DP-RMSprop         & Adaptive Clipping        & Standard Gaussian        & TBD               & TBD                    & TBD                   \\
            9                      & DP-Adam            & Adaptive Clipping        & Standard Gaussian        & TBD               & TBD                    & TBD                   \\
           \hline
        \end{tabular}
    } % End of \resizebox
    \caption{Experimental Design}  % Title of the table
    \label{tab:doe}  % Label of the table
\end{table}

\subsection{Baseline Models}\label{subsec:baseline-models}
We will compare the performance of our modified differentially private models with standard private optimizers,
including vanilla DP-SGD~\cite{Abadi_2016_DeepLearningDifferentialPrivacy}, DP-RMSprop,
and DP-Adam~\cite{zhou_2020_private_adaptive_algorithms}.
Additionally, we will benchmark against AdaClip~\cite{adaClip_2019} to evaluate the effectiveness of our automatic
clipping and noise mechanism modifications.

    \section{Related Work}\label{sec:related-work}
    Several approaches to differentially private deep learning have been explored in the literature.
    Abadi et al. \cite{Abadi_2016_DeepLearningDifferentialPrivacy} introduced DP-SGD, which has become a foundational technique for
    privacy-preserving model training.
    Our focus is on making incremental improvements to this framework by adapting it to DP-RMSprop and DP-Adam.
    Additionally, we aim to incorporate automatic clipping~\cite{bu2023automaticclippingdifferentiallyprivate}.


    \section{Timeline and Milestones}\label{sec:timeline-and-milestones}
    \begin{itemize}
        \item October 4, 2024: Best Accuracy Report 1 Due.
        \item October 18, 2024: Interim Report Due.
        \item November 8, 2024: Best Accuracy Report 2 Due.
        \item November 22, 2024: Final Report Due.
    \end{itemize}

    \section{Best Accuracy Report \#1}\label{sec:best-accuracy-report}
    \subsection{Current Implementation Overview}\label{subsec:current-implementation}
\begin{table}[!ht]

    \caption{Experimental Results}  % Title of the table

    \centering  % Center the table

    \resizebox{\textwidth}{!}{  % Resize the table to fit the width of the page

        \begin{tabular}{|c|c|c|c|c|c|c|c|}

            \hline

            \textbf{Experiment ID} & \textbf{Optimizer} & \textbf{Epochs} & \textbf{Accuracy} & \textbf{Training Time (s)} & \textbf{Privacy Cost} & \textbf{Learning Rate} & \textbf{Batch Size} \\ [0.5ex]

            \hline\hline

            1  & SGD    & 100 & 87\% & -     & -     & 0.1 & 128  \\

            2  & SGD    & 200 & 94\% & -     & -     & 0.1 & -    \\

            3  & DP-SGD & 100 & 39\% & -     & -     & 0.1 & 128  \\

            4  & DP-SGD & 30  & 40\% & 598.68 & 3    & 0.2 & 128  \\

            5  & DP-SGD & 30  & 39\% & 527.37 & 3    & 0.3 & 128  \\

            6  & DP-SGD & 30  & 37\% & 584.55 & 3    & 0.4 & 128  \\

            7  & DP-SGD & 30  & 38\% & 597.60 & 3    & 0.5 & 128  \\

            9  & DP-SGD & 30  & 35\% & 995.68 & 3    & 0.1 & 64   \\

            10 & DP-SGD & 30  & 44\% & 473.86 & 3    & 0.1 & 256  \\

            11 & DP-SGD & 30  & 44\% & 597.29 & 2.99 & 0.1 & 512  \\

            12 & DP-SGD & 30  & 42\% & 677.04 & 3    & 0.1 & 1024 \\

            13 & DP-SGD & 30  & 43\% & 519.55 & 8.01 & 0.1 & 128  \\

            14 & DP-SGD & 30  & 44\% & 627.49 & 10.01 & 0.1 & 128  \\

            15 & DP-SGD & 30  & 48\% & 553.12 & 50.04 & 0.1 & 128  \\

           \hline

        \end{tabular}

    } % End of \resizebox

    \label{tab:exp_results}  % Label of the table

\end{table}

 




\subsection{Best Observed Accuracy and Component that Attributed}\label{subsec:best-accuracy}

\subsection{Hyperparamters used}\label{subsec:hyperparameters}

\subsection{Failed Approaches}\label{subsec:failed-approaches}

\subsection{Training Methods}\label{subsec:training-methods}

    \section{Best Accuracy Report \#2}\label{sec:best-accuracy-report}
    \subsection{Current Implementation Overview}\label{subsec:current-implementation2}


\subsection{Visualizations}\label{subsec:best-accuracy-viz2}
\includegraphics[width=0.7\textwidth]{batch_size_vs_accuracy.png}
\includegraphics[width=0.7\textwidth]{learning_rate_vs_accuracy.png}
\includegraphics[width=0.7\textwidth]{Accuracy vs Noise Multiplier.png}

\subsection{Best Observed Accuracy and Components/Hyperparameters}\label{subsec:best-accuracy2}
The biggest change from the BA report \#1 was the privatization of the Lion Optimizer. To privatize, we made the gradients used in the algorithm "noisy" by introducing clipping
at the per example gradient level and then adding noise for the batch. The above results indicate the first round of testing completed with the new DP-Lion Optimizer.

We began our testing with a grid-search of possible hyperparameter combinations for batch size, learning rate, and noise multiplier. The main takeaways were that batch size,
when changed indepedently, appears to be optimal around 512 batch size. For Learning rate, it appeared that a learning rate of .001 showed the highest accuracy over a set of 5
runs. Finally, for noise multiplier, we found that raising the multiplier to 4.1 from 1.1 might provide some improvements in accuracy. We also found there to be marginal differences
between higher and lower noise multipliers when using large batch sizes.

In this suite of testing, our highest accuracy was roughly 46\%. We then turned to the original non-private Lion paper for some guidance on how they adjusted hyperparameters. They recommend 
large batch sizes and relatively small learning rates (relative to DP-Adam, DP-SGD, DP-RMSprop). They also recommend a $Beta_{1}$ value of 0.95 and a $Beta_{2}$ value of 0.98. We tested
these beta values also with increasing the epochs from 30 to 200 and achieved a peak accuracy of 54.43\%. This is the highest private accuracy we have been able to generate in our testing
yet. 

In our early testing, it seems that non-private Lion and Private Lion might prefer the same types of hyperparameters, namely large batch sizes, small learning rates, $Beta_{1}$ value of 0.95,
and a $Beta_{2}$ value of 0.98.

\begin{itemize}
    \item \textbf{Learning Rate:} We varied learning rate from 0.1 to 0.0001. The middle learning rate (0.001) yielded the highest accuracy of 45\%.
    \item \textbf{Batch Size:} We varied batch size from 64 to 2048. A batch size of 512 yielded the highest accuracy of 44\% among its peers.
    \item \textbf{Noise Multiplier:} Along with Epsilon changes, our highest accuracy, 54.43\%, also coincided with an increase in noise multiplier (10.1).
\end{itemize}

\subsection{Failed Approaches}\label{subsec:failed-approaches2}
It seems that using smaller batch sizes is not optimal for accuracy of Lion. We also notice that large learning rates like those used in DP-SGD degrade accuracy dramatically. Lastly, 
we found that $Beta_{1}$ value of 0.9 and a $Beta_{2}$ value of 0.999, which are the commonly used values for Adam, are not optimal for DP-Lion.

\subsection{Implementation Challenges}\label{subsec:implementation-challenges2}
We needed to find an implementation of Lion to be used in our model. We used the following non-private Lion implementation from the google automl repository: 
\url{https://github.com/google/automl/blob/master/lion/lion_pytorch.py}. 

From there, we needed to then pass the optimizer through the opacus make private function to return the noisy gradients optimization.
As we're looking to improve the accuracy of our models, we're needing to increase the batch size and the epochs the model runs for. This is causing us to hit a computing
limit. Some of our runs take upwards of 3600 seconds (roughly 1 hour) and can use considerable memory as well. This isn't a persistent issue, but sometimes the runs will hit memory
allocation or OOM issues on CUDA.


    \bibliographystyle{plain}
    \bibliography{references}

    \section*{GitHub Contributions}
    The code and related materials for this project are available at our GitHub repository:
    \url{https://github.com/CS8960-Privacy-Preserving-Data-Analysis/final-project}.
    Contributions, issues, and discussions are welcome.


\end{document}
