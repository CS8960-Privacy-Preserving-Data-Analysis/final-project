\subsection{Model Architecture: ResNet-20}\label{subsec:model-architecture:-resnet-20}


We propose utilizing the ResNet-20 model~\cite{Idelbayev_ResNet20} for CIFAR-10,
a standard dataset for image classification tasks.
We selected a 20-layer ResNet for its deep architecture and strength in image classification problems~\cite{DBLP:journals/corr/HeZRS15}.
20
layers should be enough depth to adequately model many features, while not encountering the higher training error encountered
with excessively \("\)deep\("\) architectures. \cite{DBLP:journals/corr/HeZRS15}

If necessary, we may modify the architecture slightly to optimize for DP compatibility.

\subsection{Non-private Optimizers to Try}\label{subsec:non-private-optimizers-to-try}
We propose implementing 4 non-private optimizers to establish baseline performance for ResNet-20 on CIFAR-10.
Three of these optimizers will be First-Order optimizers
that all build on one another.
The last optimizer that we will explore is a Second-Order optimization technique known as Cubic .

\begin{itemize}
    \item \textbf{SGD:} Standard Stochastic Gradient Descent for baseline comparison.
    This optimizer works by calculating the gradient at each data
    point and updating the model parameters with the following update rule: $\theta + \eta * \Delta g$, where $\theta$ is model parameters, $\eta$ is
    the learning rate, and $\Delta g$ is the gradient at that data point.
    \item \textbf{RMSprop:} Root Mean Square Propagation (RMSprop) builds on SGD by including the \("\)moving average\("\) factor.
    This factor functions by scaling the gradient
    each step, based on the gradient of the previous data points.
    This is done by dividing the gradient, at each model parameter update,
    by the moving average squared gradient: $(\Delta g_{t-1}^{2} * (1-\delta) + \Delta g{t}^{2}*\delta)^{\frac{1}{2}}$,
    where $\Delta g_{t-1}^{2}$ is the squared gradient average from the previous step, $\Delta g_{t}^{2}$ is
    the squared gradient of the current step, and $\delta$ is the \("\)moving average\("\) factor.  \cite{DBLP:journals/corr/abs-1807-06766,Jason_Huang_2020}
    \item \textbf{ADAM:} Adaptive Moment Estimation (ADAM) further build on RMSprop by including another \("\)moving average\("\) factor, this time for the gradient.
    In the general gradient update rule formula, instead of gradient,
    ADAM substitutes in the gradient moving average: $m_{t} = \Delta g_{t-1} * (1-\delta) + \Delta g{t}*\delta$~\cite{DBLP:journals/corr/abs-1807-06766}
\end{itemize}

\subsection{Differentially Private Optimizer: DP-SGD}\label{subsec:differentially-private-optimizer:-dp-sgd}
We propose implementing DP-SGD as our privacy-preserving algorithm.
The key components of DP-SGD are:
\begin{itemize}
    \item \textbf{Gradient Clipping:} Limits the influence of individual examples during training.
    \item \textbf{Noise Addition:} Adds noise to gradients to ensure privacy (via Opacus or TensorFlow Privacy libraries).
    \item \textbf{Privacy Accounting:} We will use Rényi Differential Privacy (RDP) for privacy budget tracking.
\end{itemize}

\subsection{Incremental Improvements}\label{subsec:incremental-improvements}
After we have privatized SGD, we propose making the following enhancements:
\begin{itemize}
    \item Adding an adaptive learning rate by incorporating the Moving Average of Gradients Squared (RMSprop)~\cite{DBLP:journals/corr/abs-1807-06766}
    \item Adding a moving average for gradient (ADAM)~\cite{DBLP:journals/corr/abs-1807-06766}
    \item Alternative noise mechanisms and their effect on utility-privacy tradeoffs. [NEED JUSTIFICATION]
    \item Adaptive gradient clipping methods that dynamically adjust the clipping threshold. [NEED JUSTIFICATION]
\end{itemize}

\subsection{Rationale for Choosing DP-SGD}\label{subsec:rationale-for-choosing-dp-sgd}
\begin{itemize}
    \item provides well-documented privacy guarantees~\cite{Abadi_2016_DeepLearningDifferentialPrivacy}
    while maintaining decent utility for image classification tasks.
    \item The addition of noise and gradient clipping help ensure $(\epsilon, \delta)$-differential privacy,
    making it ideal for sensitive applications.
    \item Previous work shows that DP-SGD, when optimized, can yield near state-of-the-art accuracy
    for differentially private models~\cite{De_2022_ScaleDP_ImageClassification}.
\end{itemize}

\subsection{Why This Approach Will Outperform Others}\label{subsec:why-this-approach-will-outperform-others}
\begin{itemize}
    \item Our approach leverages the simplicity of ResNet-20, optimized for smaller datasets like CIFAR-10,
    combined with DP-SGD, a proven differential privacy technique.
    \item By experimenting with different gradient clipping techniques and noise scales,
    we aim to find an optimal trade-off between accuracy and privacy.
    \item We will investigate modifications like adaptive clipping to enhance performance.
\end{itemize}

\subsection{Pseudocode for Non-Private SGD}\label{subsec:pseudo-code-for-non-private-sgd}
Below is a simplified pseudocode for the non-private SGD we plan to privatize:
\begin{verbatim}
    for each batch (X, y):
        pred = model(X)
        loss = loss_fn(pred, y)
        loss.backward()
        optimizer.step()
        optimizer.zero_grad()
\end{verbatim}