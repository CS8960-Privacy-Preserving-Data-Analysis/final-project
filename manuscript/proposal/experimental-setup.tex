\subsection{System Description}\label{subsec:system-description}
We will use PyTorch~\cite{pytorch_2019} for model implementation and training.
The DP-SGD~\cite{Abadi_2016_DeepLearningDifferentialPrivacy} implementation will be based on the Opacus library~\cite{opacus}.
Training will be performed on GPUs available via our departmental server csci-cuda.cs.uga.edu.

\subsection{Dataset}\label{subsec:dataset}
We will use the CIFAR-10 dataset~\cite{cifar10_dataset}, consisting of 60,000 32x32 RGB images, which is commonly used for
image classification tasks.
The dataset is built-in in PyTorch~\cite{pytorch_2019}, and we will load it using standard libraries.

\subsection{Metrics}\label{subsec:metrics}
\begin{itemize}
    \item \textbf{Training Loss/Accuracy:} Standard accuracy and training loss on CIFAR-10~\cite{cifar10_dataset}.
    \item \textbf{Privacy Cost:} We will measure $(\epsilon, \delta)$ using RDP~\cite{Mironov_2017_RenyiDP} to ensure privacy compliance.
    \item \textbf{Training Time:} Time complexity and memory usage will be tracked.
\end{itemize}

\subsection{Design of Experiments}\label{subsec:design-of-experiments}
We will be performing a series of experiments to evaluate our modified differentially private optimizers against other baseline models.
Table~\ref{tab:doe} outlines the experimental design, including the optimizer, clipping method, noise mechanism, and other metrics.

\begin{table}[!ht]
    \caption{Experimental Design}  % Title of the table
    \centering  % Center the table
    \resizebox{\textwidth}{!}{  % Resize the table to fit the width of the page
        \begin{tabular}{|c|c|c|c|c|c|c|c|}
            \hline
            \textbf{Experiment ID} & \textbf{Optimizer} & \textbf{Clipping Method} & \textbf{Noise Mechanism}   & \textbf{Accuracy} & \textbf{Training Time} & \textbf{Privacy Cost}\\ [0.5ex]
            \hline\hline
            1                      & DP-SGD             & Standard                 & Standard Gaussian        & TBD               & TBD                    & TBD                   \\
            2                      & DP-RMSprop         & Standard                 & Standard Gaussian        & TBD               & TBD                    & TBD                   \\
            3                      & DP-Adam            & Standard                 & Standard Gaussian        & TBD               & TBD                    & TBD                   \\
            4                      & DP-SGD             & Automatic Clipping       & Standard Gaussian        & TBD               & TBD                    & TBD                   \\
            5                      & DP-RMSprop         & Automatic Clipping       & Standard Gaussian        & TBD               & TBD                    & TBD                   \\
            6                      & DP-Adam            & Automatic Clipping       & Standard Gaussian        & TBD               & TBD                    & TBD                   \\
           \hline
        \end{tabular}
    } % End of \resizebox
    \label{tab:doe}  % Label of the table
\end{table}

\subsection{Baseline Models}\label{subsec:baseline-models}
We will compare the performance of our modified differentially private models with standard private optimizers,
including vanilla DP-SGD~\cite{Abadi_2016_DeepLearningDifferentialPrivacy}, DP-RMSprop,
and DP-Adam~\cite{zhou_2020_private_adaptive_algorithms}.
Additionally, we will benchmark against AdaClip~\cite{adaClip_2019} to evaluate the effectiveness of our automatic
clipping and noise mechanism modifications.