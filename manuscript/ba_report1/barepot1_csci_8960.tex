\documentclass{article}

\usepackage[preprint]{neurips_2024}
\usepackage[utf8]{inputenc}
\usepackage[T1]{fontenc}
\usepackage{hyperref}
\usepackage{url}
\usepackage{booktabs}
\usepackage{amsfonts}
\usepackage{nicefrac}
\usepackage{microtype}
\usepackage{xcolor}
\usepackage{graphicx}
\usepackage{algorithm}
\usepackage{algpseudocode}
\usepackage{amsmath}


\title{Differential Privacy in Image Classification using ResNet-20 and DP-SGD Optimization}

\author{
    Praveen Rangavajhula\\
    Department of Computer Science\\
    University of Georgia\\
    Athens, GA, 30602\\
    \texttt{praveen.rangavajhula@uga.edu} \\
    \And
    Alexander Darwiche\\
    Department of Computer Science\\
    University of Georgia\\
    Athens, GA, 30605 \\
    \texttt{alexander.darwiche@uga.edu} \\
    \And
    Deven Allen\\
    Department of Computer Science\\
    University of Georgia\\
    Athens, GA, 30605 \\
    \texttt{dca09692@uga.edu} \\
}

\begin{document}

    \maketitle

    \begin{abstract}

        This project proposes a differentially private image classification system using ResNet-20 with various
        optimizers, starting with Differentially Private Stochastic Gradient Descent (DP-SGD) as a baseline.
        We aim to make incremental improvements with additional optimization techniques, exploring both non-private
        and DP versions of optimizers, and justifying our choices based on prior work and their potential
        to outperform others.
        The project will focus on achieving competitive accuracy while satisfying privacy guarantees.
        Additionally, we are investigating ways to enhance accuracy by modifying optimizer components,
        such as gradient clipping (potentially using techniques like automatic clipping).

    \end{abstract}


    \section{Introduction}\label{sec:introduction}

    The increase in prevalence of machine learning models, especially in image classification,
    has coincided with concerns over privacy~\cite{papernot2022hyperparametertuningrenyidifferential}.
    Differential privacy (DP) specifically addresses these concerns by ensuring that models do not inadvertently leak sensitive
    information about individual data points.
    In this proposal, we will use the ResNet-20 model, which is well-suited for datasets like CIFAR-10~\cite{Idelbayev_ResNet20},
    and will implement and improve DP-SGD to achieve better privacy guarantees without significantly compromising accuracy.

    \section{Formal Description of Models Tried}\label{sec:models}
    \input{models.tex}

    \section{Related Work}\label{sec:related-work}

    \break
    \section{Preliminary Results}\label{sec:prelim-results}
    \subsection{Optimizer Details (with Hyperparameter values)}\label{subsec:optimizer-details}

\subsection{DP Details (Budget Accounting, Noise Scale, # Iterations)}\label{subsec:dp-details}
\begin{itemize}
    \item \textbf{Budget Accounting:} 
    \item \textbf{Noise Scale:}
    \item \textbf{Number of Iterations:}
\end{itemize}

 
    \break
    \section{Discussion of Results}\label{sec:results-discussion}
    \subsection{Train/Test Loss/Accuracy with varied Epochs}\label{subsec:train-testloss-accuracy}
\begin{itemize}
    \item \textbf{Accuracy/Training Loss:} Accuracy and training loss on CIFAR-10~\cite{cifar10_dataset}.
    \item \textbf{Privacy Cost:} We will measure $(\epsilon, \delta)$ using RDP~\cite{Mironov_2017_RenyiDP} to ensure privacy compliance.
    \item \textbf{Training Time:} Time complexity and memory usage will be tracked.
\end{itemize}

\subsection{Ablation Study}\label{subsec:ablation-study}

\subsection{Graphs showing effects of Hyperparameters}\label{subsec:graphs-hyperparamters}

\subsection{Tables summarizing results (with standard deviation)}\label{subsec:summary-table-results}


    \break
    \section{Learned and Plans}\label{sec:learned-and-plans}
    

    \bibliographystyle{plain}
    \bibliography{references}


    \section*{GitHub Contributions}
    The code and related materials for this project are available at our GitHub repository:
    \url{https://github.com/CS8960-Privacy-Preserving-Data-Analysis/final-project}.
    Contributions, issues, and discussions are welcome.


\end{document}
